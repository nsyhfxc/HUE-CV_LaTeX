\include{settings}

% 填写个人信息
% 学院
\newcommand{\school}{信息与电气工程学院 | School of XXXX} 

% 联系方式
\newcommand{\contact}{
    % 根据个人喜好选择字号
    % \small              % 小
    % \footnotesize       % 更小
    \scriptsize         % 再小一号
    \textcolor{white}{
        \faEnvelope \quad \href{mailto:xxxx@gmail.com}{xxxx@gmail.com}
        \hspace{4em}
        \faPhone \quad  (+86) 1xx-xxxx-xxxx
        \hspace{4em}
        \faWeixin \quad (+86) 1xx-xxxx-xxxx
        \hspace{4em}
        \faGithub \quad \href{https://github.com/xxxx}{https://github.com/xxxx}
    }
}



%%%%%%%%%%%%%%%%%%%%
% 简历正文
%%%%%%%%%%%%%%%%%%%%
\begin{document}
% 如果有多页简历,请把页眉页脚和背景复制粘贴到第二页的内容之前
    % 页眉:校标组合+学院名
    \begin{tikzpicture}[remember picture, overlay]
        \node[anchor=north, inner sep=0pt](header) at (current page.north){
            \includegraphics[width=\paperwidth]{images/header.png}
        };
        \node[anchor=west](school_logo) at (header.west){
            \hspace{0.5cm}
            \includegraphics[width=0.25\textwidth]{images/HUE_Logo1.png}
        };
        \node[anchor=east](school_name) at(header.east){
            \textcolor{white}{\textbf{\school}}
            \hspace{0.5cm}
        };
    \end{tikzpicture}
    \vspace{-3.5em}

    % 页脚,联系方式
    \begin{tikzpicture}[remember picture, overlay]
        \node[anchor=south, inner sep=0pt](footer) at (current page.south){
            \includegraphics[width=\paperwidth]{images/footer.png}
        };
        % 联系方式
        \node[anchor=center] at(footer.center){\contact};
    \end{tikzpicture}

    % 背景
    \begin{tikzpicture}[remember picture, overlay]
        \node[opacity=0.25] at(current page.center){
            \includegraphics[width=0.7\paperwidth, keepaspectratio]{images/HUE_Logo2.png}
        };
    \end{tikzpicture}

    % 个人信息
    \begin{figure}[h]
        % 左半边,信息,比例占行宽82%,可以自己调
        \begin{minipage}{0.82\textwidth}
            \section{\makebox[\widthof{\faAddressCard}][c]{\color{WHU_Blue}{\faAddressCard}}\quad 个人信息}
            \begin{tabularx}{\linewidth}{p{\widthof{出生日期:}}Xp{\widthof{政治面貌:}}X}
                姓\ \ \ \ \ \ \ \ 名: & HUE & 
                性\ \ \ \ \ \ \ \ 别: & 男/女  \\
                出生年月: & 1951年0月0日 & 
                政治面貌: & 本科高校 \\
                籍\ \ \ \ \ \ \ \ 贯: & 河北省邯郸市丛台区 & 
                最高学历: & 博士研究生 \\
                %% 想多加几行的话,就按上面的格式自行补充
                %% 想加粗的话\textbf{}
                %% 想多加几列的话,把\begin{tabularx}{\textwidth}{这里}的内容改一下,可以自己搜一下tabularx怎么用,也可以问gpt/文心一言/讯飞。
            \end{tabularx}
        \end{minipage}
    \hspace{2em}
    % 右半边,照片,比例占行宽12%,可以自己调
    % images/avatar.png 替换成你证件照的路径。
    \begin{minipage}{0.12\textwidth}
        \setlength{\fboxsep}{0pt}
        \doublebox{\includegraphics[width=\linewidth]{images/avatar.png}}
    \end{minipage}
    \end{figure}
    \vspace{-1em}

    % 教育背景
    % \faGraduationCap这类\fa开头的都是font awesome里的logo,想换成其他logo的话,可以看一下附带的fontawsome.pdf,自行替换。
    \section{\makebox[\widthof{\faGraduationCap}][c]{\color{WHU_Blue}{\faGraduationCap}}\quad 教育背景}

     %教育背景(本科生)
     \vspace{-1em}
     \begin{table}[h!]
         \begin{tabularx}{\textwidth}{XXp{\widthof{2021年 -- 预计2025年7月毕业}}}
             河北工程大学 & 电子信息工程 & 2021年 -- 预计2025年7月毕业\\
             \textbf{GPA: 4.0/4.0} & \textbf{GPA排名: 1/100} & \textbf{综测排名: 1/100} \\
         \end{tabularx}
     \end{table}

    % 教育背景(研究生)
    {\large \textbf{河北工程大学}},\textbf{本科生在读} \hfill {河北省邯郸市丛台区} \\
    \href{https://xxxx.hebeu.edu.cn/}{\underline{XX学院}},XX专业. \hfill {1970年8月-至今} \\
    {主修课程}:HUE特色课程?\ 等.

%    \vspace{0.5em}
%    {\large \textbf{河北工程大学}},硕士 \hfill {河北,邯郸} \\
%    {{园林学院}},专业:基础数学 \hfill {2010年9月-2020年6月} \\
%    \textbf{主修课程}:课程1、课程2、课程3、课程4\ 等。
%
%    \vspace{0.5em}
%    {\large \textbf{河北工程大学}},博士 \hfill {河北,邯郸} \\
%    {{研究生院}},导师:\href{导师的个人主页.site}{导师名字}\ 导师职称 \hfill {2020年9月-至今} \\
%    \textbf{研究方向}:方向1、方向2、方向3、方向4\ 等。

    % 教育背景
 \section{\makebox[\widthof{\faGraduationCap}][c]{\color{WHU_Blue}{\faGraduationCap}}\quad 个人成绩}


    % 科研著作(研究生)
    
    \textbf{2024-2025学年第一学期-XX学院-XX专业} \\
    \textbf{GPA:114515}, (1999年9月-2099年9月). \hfill 
    \textbf{专业排名:1/60}
    


    % 项目经历\科研经历\项目与教学(标题请根据需要修改)
 \section{\makebox[\widthof{\faChalkboardTeacher}][c]{\color{WHU_Blue}{\faChalkboardTeacher}}\quad 项目与学习}
    \vspace{0.5em}
    {\large{\textbf{项目名}}} \hfill {\textbf{项目是否完结}}\\
    \textbf{角色} \hfill 0000年4月-0000年5月\\
    项目简介

%    \vspace{1em}
%    {\large{\textbf{某某主题讨论班}}},主讲/参与 \hfill {2020年夏季}\\
%    主要内容:内容1,内容2,内容3\ 等。
%    
%    \vspace{1em}
%    {\large{\textbf{课程名称}}},助教 \hfill {2021年夏季}\\
%    主要内容:内容1,内容2,内容3\ 等。

    % 技能特长(标题根据个人需求修改)
 \section{\makebox[\widthof{\faWrench}][c]{\color{WHU_Blue}{\faWrench}}\quad 技能特长}
  \vspace{0.5em}
  \begin{itemize}
 	     \item 熟练使用XXXX
 	     \item 熟练使用XXXX
 	     \item 熟练使用XXXX
 	     \item 熟练掌握XXXX
 	     \item 熟练使用XXXX
 	     \item 熟练使用XXXX
 	 \end{itemize}
    
    \section{\makebox[\widthof{\faTrophy}][c]{\color{WHU_Blue}{\faTrophy}}\quad 竞赛经历}
     \vspace{-1em}
     \begin{table}[h!]
         \begin{tabularx}{\textwidth}{Xp{\widthof{第零负责人}}p{\widthof{国家级国家级国家级国家级第100名}}p{\widthof{2030年13月}}}
    		         \textbf{XXX赛} & 参加方式 & 结果 & 0000年3月 \\
    		         \textbf{XXX赛} & 参加方式 & 结果 & 0000年4月 \\
    		         \textbf{XXX赛} & 参加方式 & 结果 & 0000年5月 \\
    		         
    		         % 同理,可以自己加
    		     \end{tabularx}
    	 \end{table}
    
    
     \newpage
    % 如有需要,再加一页。可以写荣誉、竞赛等。 

    % 页眉页脚不要删。
    % % 页眉:校标组合+学院名
     \begin{tikzpicture}[remember picture, overlay]
     	\node[anchor=north, inner sep=0pt](header) at (current page.north){
     		\includegraphics[width=\paperwidth]{images/header.png}
     	};
     	\node[anchor=west](school_logo) at (header.west){
     		\hspace{0.5cm}
     		\includegraphics[width=0.25\textwidth]{images/HUE_Logo1.png}
     	};
     	\node[anchor=east](school_name) at(header.east){
     		\textcolor{white}{\textbf{\school}}
     		\hspace{0.5cm}
     	};
     \end{tikzpicture}
     \vspace{-3.5em}
     
     % 页脚,联系方式
     \begin{tikzpicture}[remember picture, overlay]
     	\node[anchor=south, inner sep=0pt](footer) at (current page.south){
     		\includegraphics[width=\paperwidth]{images/footer.png}
     	};
     	% 联系方式
     	\node[anchor=center] at(footer.center){\contact};
     \end{tikzpicture}
     
     % 背景
     \begin{tikzpicture}[remember picture, overlay]
     	\node[opacity=0.25] at(current page.center){
     		\includegraphics[width=0.7\paperwidth, keepaspectratio]{images/HUE_Logo2.png}
     	};
     \end{tikzpicture}


    % % 竞赛经历
    % \section{\makebox[\widthof{\faTrophy}][c]{\color{WHU_Blue}{\faTrophy}}\quad 竞赛经历}
    % \vspace{-1em}
    % \begin{table}[h!]
    %     \begin{tabularx}{\textwidth}{Xp{\widthof{第零负责人}}p{\widthof{国家级-第100名}}p{\widthof{2030年13月}}}
    %         \textbf{比赛1} & 第一负责人 & 国家级-第10名 & 2023年4月 \\
    %         \textbf{比赛2} & 个人参赛 & 国家级-一等奖 & 2023年8月\\
    %         \textbf{比赛3} & 个人参赛 & 省级-一等奖 & 2022年12月\\
    %         % 同理,可以自己加
    %     \end{tabularx}
    % \end{table}

    % % 技能特长
    % \section{\makebox[\widthof{\faWrench}][c]{\color{WHU_Blue}{\faWrench}}\quad 技能特长}
    % \vspace{0.5em}
    % \begin{itemize}
    %     \item 熟练使用\Cpp 、Python、Matlab编程语言。
    %     \item 熟悉Windows与Linux端开发。
    %     \item 熟练使用Tensorflow,Pytorch等深度学习框架。
    %     \item 熟练掌握\Cpp 与Python环境下OpenCV与Qt应用的开发,且熟练使用Qt Creator软件。
    %     \item 熟练使用Altium Designer与LCEDA进行封装绘制与板子设计。
    %     \item 熟练使用Keil,Arduino IDE等集成开发软件。
    %     \item 了解模式识别,强化学习,遗传算法,知识蒸馏等相关概念。
    % \end{itemize}

     % 所获荣誉
     \section{\makebox[\widthof{\faStar}][c]{\color{WHU_Blue}{\faStar}}\quad 所获荣誉}
     \vspace{-1em}
     \begin{multicols}{2}
         \begin{itemize}
             \item 某年学业先进个人
             \item 某年某奖学金某等奖
             \item 某大使
             \item 某年某奖学金某等奖
             \item 某年优秀团员称号
             \item 某年某称号
         \end{itemize}
     \end{multicols}

    % % 其他
     \section{\makebox[\widthof{\faInfo}][c]{\color{WHU_Blue}{\faInfo}}\quad 其他情况}
     \begin{itemize}
         \item 四级XXX六级XXX
         \item XXXX
         \item 熟练使用
         \item 了解
         \item 热爱
         \item 学业上
         \item 有较强的
         
     \end{itemize}

\end{document}
